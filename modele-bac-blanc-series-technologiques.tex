\documentclass[french,12pt]{article}
\usepackage[utf8]{inputenc}
\usepackage[T1]{fontenc}
\usepackage[french]{babel}
\usepackage{hyphenat}
\usepackage{parskip}
\usepackage[modulo]{lineno}
\thispagestyle{empty}
\usepackage{geometry}
\geometry{a4paper}
\geometry{hmargin=3.5cm,vmargin=1cm}
\setlength{\parindent}{1cm}
\setlength{\parskip}{0.3cm}
\hyphenchar\font=-1
\author{}
\newgeometry{bottom=2cm, top=0cm}

%VARIABLES À MODIFIER %
% Pour insérer un espace insécable : ~
% Pour forcer un passage à la ligne : \linebreak
\newcommand{\titre}{Baccalauréat blanc 2020-2021}
\newcommand{\nomProf}{M. Eyssette}
\newcommand{\classes}{TSTMG2}
\newcommand{\sujetA}{Être libre, est-ce n'obéir à aucune loi ?}
\newcommand{\sujetB}{La justice est-elle la finalité de la politique ?}
\newcommand{\sujetC}{« Méditant sur les premières et plus simples opérations de l’ame humaine, j’y crois apercevoir deux principes antérieurs à la raison, dont l’un nous intéresse ardemment à notre bien-être et à la conservation de nous-mêmes, et l’autre nous inspire une répugnance naturelle à voir périr ou souffrir tout être sensible, et principalement nos semblables. […] De cette manière on n’est point obligé de faire de l’homme un philosophe avant que d’en faire un homme~; ses devoirs envers autrui ne lui sont pas uniquement dictés par les tardives leçons de la sagesse ; et tant qu’il ne résistera point à l’impulsion intérieure de la \linebreak commisération\footnote{La commisération : la pitié, l'empathie}, il ne fera jamais du mal à un autre homme, ni même à aucun être sensible, excepté dans le cas légitime où, sa conservation se trouvant intéressée, il est obligé de se donner la préférence à lui-même. Par ce moyen on termine aussi les anciennes disputes sur la participation des animaux à la loi naturelle ; car il est clair que, dépourvus de lumières et de liberté, ils ne peuvent reconnaître cette loi ; mais tenant en quelque chose à notre nature par la sensibilité dont ils sont doués, on jugera qu’ils doivent aussi participer au droit naturel, et que l’homme est assujetti envers eux à quelque espèce de devoirs. Il semble en effet que si je suis obligé de ne faire aucun mal à mon semblable, c’est moins parce qu’il est un être raisonnable que parce qu’il est un être sensible, qualité qui, étant commune à la bête et à l’homme, doit au moins donner à l’une le droit de n’être point maltraitée inutilement par l’autre. »}
\newcommand{\prenomPhilosophe}{}
\newcommand{\nomPhilosophe}{Rousseau}
\newcommand{\titreLivre}{Discours sur l'origine et les fondements de l'inégalité parmi les hommes}
\newcommand{\precisionsReference}{}
\newcommand{\dateReference}{1755}
\newcommand{\elementsAnalyseA}{D'après Rousseau, le sens moral est-il naturel ou bien acquis socialement ?}
\newcommand{\elementsAnalyseB}{Nous n'agissons pas toujours moralement. Comment Rousseau pourrait-il expliquer cela ?}
\newcommand{\elementsAnalyseC}{Pourquoi avons-nous selon Rousseau des devoirs envers \linebreak les animaux~? Quels sont ces devoirs ?}
% \newcommand{\elementsAnalyseD}{}
\newcommand{\elementsSyntheseA}{Quelle est la question à laquelle le texte tente de répondre~?}
\newcommand{\elementsSyntheseB}{Quelles sont les étapes de l'argumentation de l'auteur ?}
\newcommand{\elementsSyntheseC}{Quelle est la thèse que défend Rousseau dans ce texte ?}
\newcommand{\commentaireA}{Ce texte soutient que la morale repose sur la sensibilité : \linebreak est-ce que cela implique que la raison n'a aucun rôle à jouer dans ce domaine~?}
\newcommand{\commentaireB}{Quel sens ce texte permet-il de donner à l'idée d'une éducation morale ?}
%FIN / VARIABLES À MODIFIER%

\title{
\titre
\\[0.5em]
\normalsize{\classes : élèves de \nomProf\\}
\date{}}

\begin{document}
\maketitle 
\vspace{-1.3cm}



\begin{center}
\textbf{Le candidat traitera, au choix, l’un des trois sujets suivants :}
\end{center}
\vspace{0.3cm}

\bigskip \noindent \textbf{Sujet 1}

\indent
\sujetA

\bigskip \noindent \textbf{Sujet 2}

\indent
\sujetB

\bigskip \noindent \textbf{Sujet 3}

\indent 
Expliquer le texte suivant :

\leftskip=1cm
\medskip \noindent
\begin{linenumbers}
\sujetC
\end{linenumbers}

\prenomPhilosophe~\textsc{\nomPhilosophe}, \textit{\titreLivre}\precisionsReference~(\dateReference)

\newpage
\vspace*{1cm}
\leftskip=0.3cm
\fboxsep=10pt
\fbox {%
\begin{minipage}{0.85\textwidth}
\textit{Rédaction de la copie}\\
Le candidat a le choix entre deux manières de rédiger l’explication de texte. Il peut :\\
\hspace{\parindent} - soit répondre dans l’ordre, de manière précise et développée, aux questions
posées (option n° 1);\\
\hspace{\parindent}  - soit suivre le développement de son choix (option n° 2).\\
Il indique son option de rédaction (option n° 1 ou option n° 2) au début de sa copie. 
\end{minipage}
}

\vspace*{1.5cm}

\leftskip=0cm
\large{\noindent \textbf{Questions de l’option n° 1}

\vspace*{0.5cm}

\noindent \textit{A – Éléments d'analyse}

\leftskip=1cm

\noindent 1) \elementsAnalyseA

\noindent 2) \elementsAnalyseB

\noindent 3) \elementsAnalyseC

% \noindent 4) \elementsAnalyseD


\vspace*{0.5cm}

\leftskip=0cm
\noindent \textit{B – Éléments de synthèse}

\leftskip=1cm
\noindent 1) \elementsSyntheseA

\noindent 2) \elementsSyntheseB

\noindent 3) \elementsSyntheseC


\vspace*{0.5cm}

\leftskip=0cm
\noindent \textit{C – Commentaire}

\leftskip=1cm

\noindent 1) \commentaireA

\noindent 2) \commentaireB

}


\end{document}